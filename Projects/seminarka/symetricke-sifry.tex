\section{Symetrické šifry}
\hyphenation{AES-NI}
Šifra s tajným klíčem je taková šifra, u nichž pro dešifrování nějaké informace je povětšinou identický jako klíč pro zašifrování. Bezpečnost a integrita dat utě chto šifer je dána tím,že příslušný klíč je znám pouze oprávněným stranám \parencite{tesar2021}.

Jedním z nejznámějších systémů založených na symetrické kryptografii jsou šifry zpracovávající data po blocích. Standard AES \enquote{(Advanced Encryption Standard)} není bloková šifra, jak je často ve veřejných publikacích zmiňováno. Tento standard totiž používá symetrickou blokovou šifru pod názvem Rijndael. Algoritmus provádí několik předem definovaných cyklů (rund), které zahrnují substituce, permutace a klíčem. Princip funkce těchto jednotlivých etap je složitý na popis a je v podstatě důležitý jen pro ty, kteří tento algoritmus implementují do svého systému \parencite {nist2023}.

V praxi se AES nejčastěji využívá k šifrování disků nebo zabezpečení síťové komunikace. Pro urychlení výpočtu tohoto algoritmu byly od roku 2010 zavedeny speciální instrukce do procesorů, například Intel \textregistered{} AES-NI, které jsou dostupné nejen na procesorech od Intelu, ale také u procesorů AMD a dalších výrobců. Tyto instrukce jsou implementovány i v mobilních zařízeních, což umožňuje efektivnější šifrování a dešifrování dat, potřebnou pro nižší spotřebu energie \textcite{abdallah2020}.

Hlavním problémem těchto systémů však stále zůstává: Jak můžeme bezpečně sdílet příslušný klíč, aniž by hrozilo jeho prozrazení třetí straně, a ověřit, že druhá strana je opravdu ta, za kterou se vydává

\subsection{Problém distribuce klíčů}
\label{sec:distribuce-klicu}
\hyphenation{Diffie-Hellman}
Distribuce klíčů je základní problémem při používání symetrických šifer, protože pro šifrování a dešifrování se používá stejný tajný klíč. Bezpečnost komunikace mezi odesílatelem a příjemcem závisí na uchování tajnosti tohoto klíče, jelikož pokud by neoprávněná strana získala onen klíč, mohla by snadno tuto komunikaci odposlouchávat.

Představme si situaci, kdy klient (například webový prohlížeč) se pokouší získat přístup k serveru a je třeba zabezpečit tento komunikační kanál. Data, která jsou mezi stranami sdílena, jsou většinou většího objemu a komunikace probíhá neustále, takže není efektivní používat asymetrickou kryptografii. K tomu, aby mohl klient a server bezpečně komunikovat, je třeba vyměnit tajný klíč tzv. secret key \mbox{\textcite{wikijs2024}}. Vzniká zde otázka: Jak bezpečně vyměnit tajný klíč přes nezabezpečený kanál, jako je transportní vrstva sítě? Jedním z efektivních řešení je právě Diffie-Hellmanův protokol, jenž je využíván napříč mnoha kryptografickými protokoly.