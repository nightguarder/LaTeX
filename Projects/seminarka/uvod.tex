\hyphenation{ko-mu-ni-kač-ních}
\section {Úvod}
Tématem této seminární práce je použití šifer v oblasti informačních a komunikačních technologií (dále jen ICT). Kryptografie, jako vědní disciplína, hraje klíčovou roli při zajišťování bezpečnosti a ochrany dat nejen v digitálním světě, kde je bezpečný přenos informací nezbytný pro každodenní komunikaci a sdílení informací. Cílem této práci je seznámit čtenáře se základními pojmy používané v dnešní kryptografii a zvýšit tak povědomí o tom kde a jak je v dnešní oblasti ICT využívána.

V první části nás práce stručně seznámí s historií a základními pojmy používanými v oblasti kryptografie, které jsou důležité k pochopení pro následující kapitoly. Nejprve bude představena Shannonova teorie informace, která tvoří teoretický základ moderní kryptografie a komunikačních systémů. Tato teorie je zásadní pro pochopení principů šifrování a bezpečné výměny dat.

Další část práce je rozdělena do tří hlavních kapitol, z nichž každá se zaměřuje na jednu z kategorií šifer používaných v současné kryptografii. První kategorie jsou symetrické šifry, též známé jako šifry s tajným klíčem. Následující kapitola se věnuje asymetrickým šifrám, kde bude podrobně vysvětlena šifra RSA, která je jedním z nejpoužívanějších šifer s veřejným klíčem \parencite{drake2024}. Poslední kategorií jsou hybridní šifry, které kombinují výhody symetrických a asymetrických šifer a jsou využívány v mnoha moderních komunikačních protokolech.

Závěrem se práce bude věnovat oblasti postkvantové kryptografie, která se zabývá vývojem šifrovacích algoritmů odolných vůči kvantovým počítačům. V této části se seznámíme s aktuálními výzvami a vývojem v oblasti kryptografie, včetně Shorova algoritmu, který představuje hrozbu pro současné asymetrické šifry, zejména RSA.
\newpage