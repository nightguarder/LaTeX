\documentclass[12pt,a4paper,oneside]{article}

% Polyglossia 
\usepackage{polyglossia}
\setdefaultlanguage{czech}
\usepackage{fontspec}
\setmainfont{Times New Roman} % nebo jiný Unicode font

% Matematické znaky
\usepackage{amsmath}        
\usepackage{amsfonts}       
\usepackage{amsthm}         
\usepackage{bm}             

% Obrazky
\usepackage{graphicx}
\newcommand{\FIGURES}{./res} %cesta k adresáři s obrázky

% Bibtex + Biber 
\usepackage[backend=biber,style=numeric,style=iso-authoryear, autolang=other]{biblatex}
\addbibresource{references.bib}

% Typograficke minimum
\ifx
\uv\undefined
\newcommand{\uv}[1]{\enquote{#1}} % Definice makra pro sazbu českých uvozovek
\fi

\usepackage{parskip} % Vertical space between paragraphs
\setlength{\parindent}{1.5em} % First line indentation

% Nastaveni odkazu
\usepackage[unicode]{hyperref} % generovani odkazu v PDF
\hypersetup{pdftitle=Název práce, 
            pdfauthor=Jméno Příjmení
            ps2pdf,
            colorlinks=false,
            urlcolor=blue,
            pdfstartview=FitH,
            pdfpagemode=UseOutlines,
            pdfnewwindow,
            breaklinks                      %% zajistí, aby se dlouhé hyperodkazy mohly lámat přes více řádků
}
\usepackage[nottoc]{tocbibind} % Literatura v obsahu


\begin{document}

%Titulní strana
\title{Využití šifer v informační komunikační technologii}
\author{Cyril Steger}
\date{12.2.2024}
\pagestyle{empty}
\begin{center}
    
    {\large Technická univerzita v Liberci}

    \medskip
    {\large Ekonomická fakulta}
    %logo
    \vfill
    \centerline{\mbox{\includegraphics[width=55mm]{\FIGURES/tul-ef_symbol_colour_RGB.png}}} %logo
    \vfill
    \vspace{5mm}
    %title cz
    {\LARGE\bfseries Využití šifer v informační komunikační technologii \\}
    %title en
    \vspace{5mm}
    {\large Usage of Cryptography in information communication technology \\}

    \vfill
    \vspace{5mm}
    {\large Cyril Steger}
    
    \vfill
    \begin{tabular}{rl}
        Vedoucí seminární práce: &  Ing. David Kubát, Ph.D.\\   %% Jméno a příjmení s~tituly 
        \noalign{\vspace{2mm}}
        Studijní program: & Systémové inženýrství a informatika\\
        \noalign{\vspace{2mm}}
        Studijní obor: & Navazující studium prezenční\\
        \noalign{\vspace{2mm}}
    \end{tabular}
    \medskip

    \vfill
    {\large Liberec 2024}
    
\end{center}
%Obsah
\newpage
\pagestyle{plain}
\setcounter{page}{1}

\tableofcontents


%Seznam literatury
\newpage
\section*{Seznam literatury}
\addcontentsline{toc}{section}{Seznam literatury}
\printbibliography[heading=none]

%Seznam obrázků
\newpage
\listoffigures  

\end{document}
