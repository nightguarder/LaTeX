\documentclass[12pt,a4paper,oneside]{article}

% Polyglossia 
\usepackage{polyglossia}
\setdefaultlanguage{czech}
\usepackage{fontspec}
\setmainfont{Times New Roman} % nebo jiný Unicode font

% Matematické znaky
\usepackage{amsmath}        
\usepackage{amsfonts}       
\usepackage{amsthm}         
\usepackage{bm}


% Obrazky
\usepackage{graphicx}
\newcommand{\FIGURES}{./res} %cesta k adresáři s obrázky

% Bibtex + Biber 
\usepackage[backend=biber,style=iso-authoryear, autolang=other]{biblatex}
\addbibresource{references.bib}

% Typograficke minimum

\usepackage{csquotes} % doporučuje se pro biblatex
\usepackage{microtype}
\usepackage{parskip} % odstavce bez odsazení, ale s odstupem

% Nastaveni odkazu
\usepackage[unicode]{hyperref} % generovani odkazu v PDF
\hypersetup{pdftitle=Název práce, 
            pdfauthor=Jméno Příjmení,
            colorlinks=false,
            urlcolor=blue,
            pdfstartview=FitH,
            pdfpagemode=UseOutlines,
            pdfnewwindow,
            breaklinks  %% zajistí, aby se dlouhé hyperodkazy mohly lámat přes více řádků
}
\usepackage[nottoc]{tocbibind} % Automatically adds the bibliography and/or the index and/or the contents, etc., to the Table of Contents listing.

\begin{document}

%Titulní strana
\title{Využití šifer v informační komunikační technologii}
\author{Cyril Steger}
\date{12.2.2024}
\pagestyle{empty}
\begin{center}
    
    {\large Technická univerzita v Liberci}

    \medskip
    {\large Ekonomická fakulta}
    %logo
    \vfill
    \centerline{\mbox{\includegraphics[width=55mm]{\FIGURES/tul-ef_symbol_colour_RGB.png}}} %logo
    \vfill
    \vspace{5mm}
    %title cz
    {\LARGE\bfseries Využití šifer v informační komunikační technologii \\}
    %title en
    \vspace{5mm}
    {\large Usage of Cryptography in information communication technology \\}

    \vfill
    \vspace{5mm}
    {\large Cyril Steger}
    
    \vfill
    \begin{tabular}{rl}
        Vedoucí seminární práce: &  Ing. David Kubát, Ph.D.\\   %% Jméno a příjmení s~tituly 
        \noalign{\vspace{2mm}}
        Studijní program: & Systémové inženýrství a informatika\\
        \noalign{\vspace{2mm}}
        Studijní obor: & Navazující studium prezenční\\
        \noalign{\vspace{2mm}}
    \end{tabular}
    \medskip

    \vfill
    {\large Liberec 2024}
    
\end{center}
%Obsah
\newpage
\pagestyle{plain}
\setcounter{page}{1}

\tableofcontents

%Kapitoly
%Uvod
\hyphenation{ko-mu-ni-kač-ních}
\section {Úvod}
Tématem této seminární práce je použití šifer v oblasti informačních a komunikačních technologií (dále jen ICT). Kryptografie, jako vědní disciplína, hraje klíčovou roli při zajišťování bezpečnosti a ochrany dat nejen v digitálním světě, kde je bezpečný přenos informací nezbytný pro každodenní komunikaci a sdílení informací. Cílem této práci je seznámit čtenáře se základními pojmy používané v dnešní kryptografii a zvýšit tak povědomí o tom kde a jak je v dnešní oblasti ICT využívána.

V první části nás práce stručně seznámí s historií a základními pojmy používanými v oblasti kryptografie, které jsou důležité k pochopení pro následující kapitoly. Nejprve bude představena Shannonova teorie informace, která tvoří teoretický základ moderní kryptografie a komunikačních systémů. Tato teorie je zásadní pro pochopení principů šifrování a bezpečné výměny dat.

Další část práce je rozdělena do tří hlavních kapitol, z nichž každá se zaměřuje na jednu z kategorií šifer používaných v současné kryptografii. První kategorie jsou symetrické šifry, též známé jako šifry s tajným klíčem. Následující kapitola se věnuje asymetrickým šifrám, kde bude podrobně vysvětlena šifra RSA, která je jedním z nejpoužívanějších šifer s veřejným klíčem \parencite{drake2024}. Poslední kategorií jsou hybridní šifry, které kombinují výhody symetrických a asymetrických šifer a jsou využívány v mnoha moderních komunikačních protokolech.

Závěrem se práce bude věnovat oblasti postkvantové kryptografie, která se zabývá vývojem šifrovacích algoritmů odolných vůči kvantovým počítačům. V této části se seznámíme s aktuálními výzvami a vývojem v oblasti kryptografie, včetně Shorova algoritmu, který představuje hrozbu pro současné asymetrické šifry, zejména RSA.
\newpage



%Seznam literatury
\newpage
\section*{Seznam literatury}
\addcontentsline{toc}{section}{Seznam literatury}
\printbibliography[heading=none]

%Seznam obrázků
\newpage
\listoffigures  

\end{document}
